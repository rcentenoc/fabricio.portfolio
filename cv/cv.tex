\documentclass[10pt, letterpaper]{article}
% =======================
% Packages y configuración base
% =======================
% Packages:
\usepackage[
    ignoreheadfoot, % set margins without considering header and footer              
    top=2 cm, % seperation between body and page edge from the top
    bottom=2 cm, % seperation between body and page edge from the bottom
    left=2 cm, % seperation between body and page edge from the left
    right=2 cm, % seperation between body and page edge from the right
    footskip=1.0 cm, % seperation between body and footer
    % showframe % for debugging 
]{geometry} % for adjusting page geometry
\usepackage{iftex}
\ifPDFTeX
    \input{glyphtounicode}
    \pdfgentounicode=1
    \usepackage[T1]{fontenc}
    \usepackage[utf8]{inputenc}
    \usepackage{lmodern}
\fi
\usepackage[spanish,es-noshorthands]{babel}
\usepackage{microtype}
\usepackage{titlesec} % for customizing section titles
\usepackage{tabularx} % for making tables with fixed width columns
\usepackage{array} % tabularx requires this
\usepackage[dvipsnames]{xcolor} % for coloring text
\definecolor{primaryColor}{RGB}{0, 0, 0} % define primary color
\usepackage{enumitem} % for customizing lists
\usepackage{fontawesome5} % for using icons
\usepackage{amsmath} % for math
\usepackage{graphicx}
\usepackage[
    pdftitle={FAB's CV},
    pdfauthor={Fab},
    pdfcreator={LaTeX with RenderCV},
    colorlinks=true,
    urlcolor=primaryColor
]{hyperref} % for links, metadata and bookmarks
\usepackage[pscoord]{eso-pic} % for floating text on the page
\usepackage{calc} % for calculating lengths
\usepackage{bookmark} % for bookmarks
\usepackage{lastpage} % for getting the total number of pages
\usepackage{changepage} % for one column entries (adjustwidth environment)
\usepackage{paracol} % for two and three column entries
\usepackage{ifthen} % for conditional statements
\usepackage{needspace} % for avoiding page brake right after the section title

% Ensure that generate pdf is machine readable/ATS parsable:
% \ifPDFTeX
%     \input{glyphtounicode}
%     \pdfgentounicode=1
%     \usepackage[T1]{fontenc}
%     \usepackage[utf8]{inputenc}
%     \usepackage{lmodern}
% \fi

\usepackage{charter}

\usepackage{tikz}
\usetikzlibrary{shadows,calc}
% QR en esquina superior izquierda (top-left), con estilo
% Uso: \TopLeftQR[<tamano>]{<URL>}  --> p.ej. \TopLeftQR[2.1cm]{https://tusitio.com}
\newcommand{\TopLeftQR}[2][2cm]{%
  \AddToShipoutPictureFG*{%
    % Coordenadas en puntos de la página (origen: esquina inferior izquierda)
    % X = 2cm desde el borde izquierdo; Y = altura de página - 2cm (alineado con tus márgenes)
    \put(\LenToUnit{0.3cm},\LenToUnit{\paperheight-2.5cm}){%
      \makebox[0pt][l]{\href{#2}{\includegraphics[width=#1]{one.png}}}%
    }%
  }%
}

% =======================
% Ajustes globales
% =======================
% Some settings:
\raggedright
\AtBeginEnvironment{adjustwidth}{\partopsep0pt} % remove space before adjustwidth environment
\pagestyle{empty} % no header or footer
\setcounter{secnumdepth}{0} % no section numbering
\setlength{\parindent}{0pt} % no indentation
\setlength{\topskip}{0pt} % no top skip
\setlength{\columnsep}{0.15cm} % set column seperation
\pagenumbering{gobble} % no page numbering

\titleformat{\section}
    {\needspace{4\baselineskip}\bfseries\large}
    % {\needspace{4\baselineskip}\large\bfseries\scshape}
    {}{0pt}{}
    [\vspace{2pt}\titlerule]

\titlespacing{\section}{
    % left space:
    -1pt
}{
    % top space:
    0.3 cm
}{
    % bottom space:
    0.2 cm
} % section title spacing

\renewcommand\labelitemi{$\vcenter{\hbox{\small$\bullet$}}$} % custom bullet points
\newenvironment{highlights}{
    \begin{itemize}[
        topsep=0.10 cm,
        parsep=0.10 cm,
        partopsep=0pt,
        itemsep=0pt,
        leftmargin=0 cm + 10pt
    ]
}{
    \end{itemize}
} % new environment for highlights


\newenvironment{highlightsforbulletentries}{
    \begin{itemize}[
        topsep=0.10 cm,
        parsep=0.10 cm,
        partopsep=0pt,
        itemsep=0pt,
        leftmargin=10pt
    ]
}{
    \end{itemize}
} % new environment for highlights for bullet entries

\newenvironment{onecolentry}{
    \begin{adjustwidth}{
        0 cm + 0.00001 cm
    }{
        0 cm + 0.00001 cm
    }
}{
    \end{adjustwidth}
} % new environment for one column entries

\newenvironment{twocolentry}[2][]{
    \onecolentry
    \def\secondColumn{#2}
    \setcolumnwidth{\fill, 4.5 cm}
    \begin{paracol}{2}
}{
    \switchcolumn \raggedleft \secondColumn
    \end{paracol}
    \endonecolentry
} % new environment for two column entries

\newenvironment{threecolentry}[3][]{
    \onecolentry
    \def\thirdColumn{#3}
    \setcolumnwidth{, \fill, 4.5 cm}
    \begin{paracol}{3}
    {\raggedright #2} \switchcolumn
}{
    \switchcolumn \raggedleft \thirdColumn
    \end{paracol}
    \endonecolentry
} % new environment for three column entries

\newenvironment{header}{
    \setlength{\topsep}{0pt}\par\kern\topsep\centering\linespread{1.4}
}{
    \par\kern\topsep
} % new environment for the header

% Sello "Actualizado: hoy" en esquina sup. derecha
\newcommand{\lastupdated}{%
  \AddToShipoutPictureFG*{%
    \put(\LenToUnit{\paperwidth-2cm},\LenToUnit{\paperheight-1.0cm}){%
      \makebox[0pt][r]{\small\color{gray}\textit{Actualizado: \today}}%
    }%
  }%
}

% save the original href command in a new command:
\let\hrefWithoutArrow\href
\newcommand{\AND}{\unskip
  \cleaders\copy\ANDbox\hskip\wd\ANDbox \ignorespaces}
\newsavebox\ANDbox
\sbox\ANDbox{$|$}


% new command for external links:

% /////////////////////////////////////////////////////////////////////////////////////////////////////////

\begin{document}
\lastupdated
\TopLeftQR[2.3cm]{https://fabriciocc.vercel.app/} % o tu URL personal
    \begin{header}
        \fontsize{24pt}{26pt}\selectfont
        \textbf{Ronald Fabricio Centeno Cárdenas}

        \vspace{5 pt}

        \normalsize
        \mbox{\faEnvelope[regular]\;\hrefWithoutArrow{mailto:efren609@duck.com}{efren609@duck.com}}%
        \kern 5.0 pt%
        \AND%
        \kern 5.0 pt%
        \mbox{\hrefWithoutArrow{https://github.com/rcentenoc}\faGithub\;\hrefWithoutArrow{https://github.com/rcentenoc}{github.com/rcentenoc}}%
        \kern 5.0 pt%
        \AND%
        \kern 5.0 pt%
        \mbox{\hrefWithoutArrow{https://www.linkedin.com/in/r-fabricio}\faLinkedinIn\;\hrefWithoutArrow{https://www.linkedin.com/in/r-fabricio}{linkedin.com/r-fabricio}}%
        \kern 5.0 pt%
        \AND%
        \kern 5.0 pt%
        \mbox{\hrefWithoutArrow{https://api.whatsapp.com/send/?phone=51932001792}\faWhatsapp\;\hrefWithoutArrow{https://api.whatsapp.com/send/?phone=51932001792}{+51 932 001 792}}%
        \kern 5.0 pt%
        
        \mbox{Arequipa, Perú}%
    \end{header}

    \vspace{5 pt - 0.3 cm}

% /////////////////////////////////////////////////////////////////////////////////////////////////////////

    \section{}        
        \begin{onecolentry}
            \begin{center}
            % Ingeniero de Software con más de 4 años de experiencia como desarrollador Fullstack, especializado en sistemas escalables con Java, TypeScript, Node.js y Spring Boot. Con experiencia en bases de datos como MongoDB, MySQL y PostgreSQL, y herramientas de monitoreo como Morgan. Autodidacta y con experiencia en equipos ágiles bajo Scrum.  Busco contribuir en proyectos como desarrollador backend o en roles FullStack, aprovechando mi conocimiento en tecnologías modernas y mi pasión por la optimización de procesos.
            Ingeniero de software Full-Stack con más de 4 años de experiencia, enfocado a construir aplicaciones modernas y potenciarlas con IA. Cuento con experiencia en diferentes frameworks y stacks de desarrollo de software, además de desarrollo movil. Actualmente, incursiono en el diseño e integración de sistemas RAG y flujos basados en MCP para aumentar relevancia y automatización; también he usado n8n para orquestar procesos. Busco aportar valor en equipos que escalen productos de IA con foco en calidad, seguridad y observabilidad.
            
            \end{center}
        \end{onecolentry}

% /////////////////////////////////////////////////////////////////////////////////////////////////////////

    \section{Experiencia}
        
        \begin{twocolentry}{
            Feb, 2025 – Actualmente
        }
            \textbf{Desarrollador FullStack}, Khipucode

        \vspace{0.10 cm}
            \begin{highlights}
                \vspace{0.1 cm}
                \item Diseño y desarrollo de servicios backend (Node.js/NestJS), exposición de APIs REST y GraphQL, tiempo real con Socket.IO y mensajería/event streaming con Apache Kafka.
                \item Integración de IA con Gemini para extracción estructurada de texto desde documentos; automatización de reportes (PDF/Excel) con plantillas Carbone.
                \item Desarrollo de una API que obtiene y normaliza precios oficiales de oro y plata (scraping responsable, validaciones, scheduler y observabilidad), además de evaluar e incorporar nuevas tecnologías.
            \end{highlights}
        \end{twocolentry}


        \vspace{0.3 cm}
        
        \begin{twocolentry}{
            Oct, 2024 – Feb, 2025
        }
            \textbf{Tesista}, CEEP-Eficiencia energética

        \vspace{0.10 cm}
            \begin{highlights}
                \vspace{0.1 cm}
                \item Desarrollo de un procesador universal de datos basado en binarización, enfocado en reducir la frecuencia de envío y mejorar la precisión en sistemas IoT.
                \item Implementación de técnicas de machine learning con Java, Spring Boot, MySQL y AWS.
            \end{highlights}
            \end{twocolentry}


        \vspace{0.3 cm}

        \begin{twocolentry}{
            May, 2023 – Ago 2023
        }
            \textbf{Practicante de Desarrollo Web}, FAMAI SEAL JET S.A.C.

        \vspace{0.10 cm}
            \begin{highlights}
                \vspace{0.1 cm}
                \item Desarrollo de un sistema de control de asistencia para empleados utilizando React.js, Node.js y SQL Server.
                \item Gestión de mantenimiento de software y hardware bajo estándares ISO 9001, y administración de sistemas empresariales como CPanel y Outlook.
            \end{highlights}
        \end{twocolentry}

        \vspace{0.3 cm}
        \begin{twocolentry}{
            Feb, 2022 – Abr 2024
        }
            \textbf{Desarrollador BackEnd}, TDX PERÚ S.A.C.

        \vspace{0.10 cm}
            \begin{highlights}
                \vspace{0.1 cm}
                \item Desarrollo de aplicativo móvil en React Native para asistencia en traducciones, integrando API de DeepL para traducción automática en tiempo real.
                \item Refactorización y mantenimiento de sistema de gestión documental desarrollado con Spring Boot, MySQL, MongoDB y Redis, enfocándome en la reducción de deuda técnica y mejora del rendimiento.
                \item Optimización de flujos backend y mejora de la arquitectura existente, asegurando escalabilidad y estabilidad del sistema en entornos productivos.
            \end{highlights}
            \end{twocolentry}


% /////////////////////////////////////////////////////////////////////////////////////////////////////////
    
   
    \section{Educación}
       
        \begin{twocolentry}{
            Feb 2018 – Dic 2023
        }
            \textbf{Universidad La Salle, Arequipa}, Ingeniería de Software
        \end{twocolentry}

        \vspace{0.10 cm}
        \begin{onecolentry}
            \begin{highlights}
                \item Bachiller en Ingeniería de Software.
                \item \textbf{Coursework:} Arquitectura de software, Inteligencia Artificial, Desarrollo de compiladores, Desarrollo de Software, Calidad y Elicitación de requisitos de Software. 
            \end{highlights}
        \end{onecolentry}
        \vspace{0.30 cm}

        \begin{twocolentry}{
            Ago 2023 - Ago 2024
        }
            \textbf{Programa de Oracle One}, Desarrollo BackEnd\end{twocolentry}

        \vspace{0.10 cm}
        \begin{onecolentry}
            \begin{highlights}
                \item \href{https://app.aluracursos.com/user/efren609/fullCertificate/124b1c05ec9e67541c81dad40a470321}{Certificación del programa Oracle One: Backend con Java y Springboot.} 
                \item \textbf{Coursework:} Oracle Cloud Infrastructure, Java, Springboot, Mysql.
            \end{highlights}
        \end{onecolentry}
        
        \vspace{0.30 cm}

        \begin{twocolentry}{
            Ene 2022 – Feb 2024
        }
            \textbf{Platzi}, Escuela de JavaScript\end{twocolentry}

        \vspace{0.10 cm}
        \begin{onecolentry}
            \begin{highlights}
                \item \href{https://platzi.com/p/fabriciocendecar/ruta/100-javascript-full-stack/diploma/detalle/}{Certificación FullStack Developer con JavaScript.}
                \item \textbf{Coursework:} Backend con Node.js, Full Stack con Next.js, Desarrollo de Apps con React Native, Frontend con React.js.
            \end{highlights}
        \end{onecolentry}

% /////////////////////////////////////////////////////////////////////////////////////////////////////////

       
    \section{Publicaciones}



        
        \begin{samepage}
            \begin{twocolentry}{
                Feb 2022
            }
                \textbf{Ranking of tutorials on YouTube based on the analysis of feelings made to their comments}
            \end{twocolentry}

            \vspace{0.10 cm}
            
            \begin{onecolentry}
                

                \vspace{0.10 cm}
                
        \href{https://doi.org/10.48168/innosoft.s9.a66}{10.48168/innosoft.s9.a66}
        \end{onecolentry}
        \end{samepage}


    
    % \section{Projects}
        
    %     \begin{twocolentry}{
    %         \href{https://github.com/sinaatalay/rendercv}{github.com/name/repo}
    %     }
    %         \textbf{Multi-User Drawing Tool}\end{twocolentry}

    %     \vspace{0.10 cm}
    %     \begin{onecolentry}
    %         \begin{highlights}
    %             \item Developed an electronic classroom where multiple users can simultaneously view and draw on a "chalkboard" with each person's edits synchronized
    %             \item Tools Used: C++, MFC
    %         \end{highlights}
    %     \end{onecolentry}


    %     \vspace{0.2 cm}

    %     \begin{twocolentry}{
    %         \href{https://github.com/sinaatalay/rendercv}{github.com/name/repo}
    %     }
    %         \textbf{Synchronized Desktop Calendar}\end{twocolentry}

    %     \vspace{0.10 cm}
    %     \begin{onecolentry}
    %         \begin{highlights}
    %             \item Developed a desktop calendar with globally shared and synchronized calendars, allowing users to schedule meetings with other users
    %             \item Tools Used: C\#, .NET, SQL, XML
    %         \end{highlights}
    %     \end{onecolentry}


    %     \vspace{0.2 cm}

    %     \begin{twocolentry}{
    %         2002
    %     }
    %         \textbf{Custom Operating System}\end{twocolentry}

    %     \vspace{0.10 cm}
    %     \begin{onecolentry}
    %         \begin{highlights}
    %             \item Built a UNIX-style OS with a scheduler, file system, text editor, and calculator
    %             \item Tools Used: C
    %         \end{highlights}
    %     \end{onecolentry}


    %     \includegraphics[width=2cm]{one.png}

    
    % \section{Technologies}



        
    %     \begin{onecolentry}
    %         \textbf{Languages:} C++, C, Java, Objective-C, C\#, SQL, JavaScript
    %     \end{onecolentry}

    %     \vspace{0.2 cm}

    %     \begin{onecolentry}
    %         \textbf{Technologies:} .NET, Microsoft SQL Server, XCode, Interface Builder
    %     \end{onecolentry}


\section{Tecnologías}
\begin{onecolentry}
  \textbf{Lenguajes/Frameworks:} Java, TypeScript/JavaScript, Node.js, NestJS, Spring Boot, React, Next.js, Vue, Flutter. \\
  \textbf{Datos/Infra:} PostgreSQL, MySQL, MongoDB, Kafka, Socket.IO, AWS. \\
  \textbf{IA/Automatización:} RAG, MCP, Gemini, n8n, Carbone. \\
\end{onecolentry}

    

\end{document}